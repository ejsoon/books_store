\documentclass[20pt,a4paper]{article}
\usepackage[top=0.6in,bottom=0.6in,left=1.0in,right=1.0in]{geometry}

\usepackage{xeCJK}
\setCJKmainfont[RawFeature={vertical,size=20pt}]{TW-Kai}

\usepackage{setspace}
\setstretch{1.5}

\pagestyle{empty}

\begin{document}
\begin{Huge}
︽三字經︾

宋 王應麟 著
\\

\end{Huge}
\begin{huge}
人之初,性本善。性相近,習相遠。

苟不教,性乃遷。教之道,貴以專。

昔孟母,擇鄰處。子不學,斷機杼。

竇燕山,有義方,教五子,名俱揚。

養不教,父之過。教不嚴,師之惰。

子不學,非所宜。幼不學,老何為?

玉不琢,不成器。人不學,不知義。

為人子,方少時,親師友,習禮儀。

香九齡,能溫席。孝於親,所當執。

融四歲,能讓梨。弟于長,宜先知。
\newpage

首孝弟,次見聞。知某數,識某文。

一而十,十而百,百而千,千而萬。

三才者,天地人。三光者,日月星。

三綱者:君臣義,父子親,夫婦順。

曰春夏,曰秋冬,此四時,運不窮。

曰南北,曰西東,此四方,應乎中。

曰水火,木金土,此五行,本乎數。

曰仁義,禮智信,此五常,不容紊。

稻粱菽,麥黍稷,此六穀,人所食。

馬牛羊,雞犬豕,此六畜,人所飼。

曰喜怒,曰哀懼,愛惡欲,七情具。

匏土革,木石金,與絲竹,乃八音。

高曾祖,父而身,身而子,子而孫。

自子孫,至玄曾,乃九族,人之倫。

父子恩,夫婦從,兄則友,弟則恭,

長幼序,友與朋,君則敬,臣則忠,

此十義,人所同。
\newpage

凡訓蒙,須講究。詳訓詁,明句讀。

為學者,必有初。小學終,至四書。

論語者,二十篇,群弟子,記善言。

孟子者,七篇止,講道德,說仁義。

作中庸,子思筆,中不偏,庸不易。

作大學,乃曾子,自修齊,至平治。

孝經通,四書熟,如六經,始可讀。

詩書易,禮春秋,號六經,當講求。

有連山,有歸藏,有周易,三易詳。

有典謨,有訓誥,有誓命,書之奧。

我周公,作周禮,著六官,存治體。

大小戴,註禮記,述聖言,禮樂備。

曰國風,曰雅頌,號四詩,當諷詠。

詩既亡,春秋作,寓褒貶,別善惡。

三傳者,有公羊,有左氏,有穀梁。

經既明,方讀子,撮其要,記其事。

五子者,有荀揚,文中子,及老莊。
\clearpage

\end{huge}
\begin{LARGE}
經子通,讀諸史。考世系,知終始。

自羲農,至黃帝,號三皇,居上世。

唐有虞,號二帝,相揖遜,稱盛世。

夏有禹,商有湯,周文武,稱三王。

夏傳子,家天下,四百載,遷夏社。

湯伐夏,國號商,六百載,至紂亡。

周武王,始誅紂,八百載,最長久。

周轍東,王綱墜,逞干戈,尚遊說。

始春秋,終戰國,五霸強,七雄出。

嬴秦氏,始兼并,傳二世,楚漢爭。

高祖興,漢業建,至孝平,王莽篡。

光武興,為東漢,四百年,終於獻。

魏蜀吳,爭漢鼎,號三國,迄兩晉。

宋齊繼,梁陳承,為南朝,都金陵。

北元魏,分東西,宇文周,與高齊。

迨至隋,一土宇,不再傳,失統緒。

唐高祖,起義師,除隋亂,創國基。

二十傳,三百載,梁滅之,國乃改。

梁唐晉,及漢周,稱五代,皆有由。

炎宋興,受周禪,十八傳,南北混。

十七史,全在茲,載治亂,知興衰。

讀史者,考實錄,通古今,若親目。

口而誦,心而惟,朝於斯,夕於斯。
\newpage
\end{LARGE}
\begin{huge}

昔仲尼,師項橐。古聖賢,尚勤學。

趙中令,讀魯論。彼既仕,學且勤。

披蒲編,削竹簡。彼無書,且知勉。

頭懸樑,錐刺股。彼不教,自勤苦。

如囊螢,如映雪。家雖貧,學不輟。

如負薪,如掛角。身雖勞,猶苦卓。

蘇老泉,二十七,始發憤,讀書籍。

彼既老,猶悔遲。爾小生,宜早思。

若梁顥,八十二,對大廷,魁多士。

彼既成,眾稱異。爾小生,宜立志。

瑩八歲,能詠詩。泌七歲,能賦棋。

彼穎悟,人稱奇。爾幼學,當效之。

蔡文姬,能辨琴。謝道韞,能詠吟。

彼女子,且聰敏。爾男子,當自警。

唐劉晏,方七歲,舉神童,作正字。

彼雖幼,身已仕。爾幼學,勉而致。

有為者,亦若是。
\newpage

犬守夜,雞司晨。苟不學,曷為人?

蠶吐絲,蜂釀蜜。人不學,不如物。

幼而學,壯而行,上致君,下澤民。

揚名聲,顯父母,光于前,裕於後。

人遺子,金滿籝。我教子,惟一經。

勤有功,戲無益。戒之哉!宜勉力。

︵完︶

\end{huge}
\end{document}
