\documentclass{article}
\usepackage{fontspec}
\setmainfont{TW-Sung}
\XeTeXlinebreaklocale "zh"
\XeTeXlinebreakskip = 0pt plus 1pt 
\title{倉頡字源}
\author{尹杉 | Ejsoon Y}
\date{\today}
\begin{document}
\maketitle 
本文以“非商業、保留作者名字、勿更改內容”進行分享


\LaTeX\


\section{<前言>}

倉頡,遠古造字官神,“首有四目,通於神明,仰觀奎星圜曲之勢,俯察龜文鳥跡之像,採乎眾美,合而為字”。祂書文畫彡之壯舉,“天為雨粟,鬼為夜哭,龍為潛藏”。\\
“蓋文字者,經藝之本,王政之始,前人所以垂後,後人所以識古。”吾嘗覽古通今,一則自悅,二則養慧。<<大學>>有云:“格物,致知。”文字者物乎?知乎?通其二者也。必先以溯源識義,而後得生慧也。\\
\\
倉頡輸入法,係朱邦復發明於西元一九七六年。在發表了二代之後,朱邦復先生總結前期經驗,選取<<康熙字典>>四萬字,改良成爲“趨向完整”的第三代倉頡。此後的四、五代,只是在第三代的基礎上稍加修補,其規則和字母大致不相偏離。今吾所論述倉頡者,爲第三代倉頡輸入法。\\
朱邦復在始創倉頡之時,便將字碼聯繫到“形辨音義”之上,吾深明其理,爲之欽佩不已。恰逢近日所遇友人疑慮頗偏,亟須行表以正聽。謹將此文作一簡覆。\\
\\
因時間倉促,疏漏難免,還待多方批評指正!\\



\section{<目錄>}

\tableofcontents

%\subsection{前言}
%\subsection{一.楷書構形與字源}
%\subsection{二.倉頡字母之來源}
%\subsection{三.漢字舉例與解說}
%\subsection{四.輸入法對比評說}
%\subsection{結語}



\section{<一.楷書構形與字源>}

由於倉頡是純粹以楷書形構組合對漢字進行取碼,欲知倉頡所能推字源之極限,必先道明楷書與字源之關聯。\\
\\
漢字經歷幾千年的發展,其中一脈由甲骨文,金文,篆體,隸書,終演化爲楷書。楷書不論在手寫還是印刷,其實用性,標準性和觀賞性都無可替代。從漢代至今二千餘年,朝代興亡,風風雨雨,楷書之正體地位卻未曾動搖。\\
\\
楷書之於字源,二者在一定程度上既有承襲,亦有脫離。\\

\subsection{絕大多數的楷書字形可以承襲字源字義}
例如“疌妻”二字,篆體均有“以手持艸”的字形“屮肀”,演化至楷書後,除了因楷書橫平豎直的特點而改變其走勢之外,其組合構形基本未變。“止/女”在楷書體系內亦是與字源能壹一映射的關係。\\
由於可舉之例甚眾,在此不多贅述。\\

\subsection{少部分構件發生形變}
舉例:\\
“心”在左邊就會變成“忄”。\\
“人”在不同位置可能變化爲“亻儿⺈”。\\
“阜”在左邊、“邑”置於右會變爲“阝”。\\
“萬”象百足蟲形,楷書把其足之部分訛作“艸”。\\
“泉”本作爲水之源流成川形,楷書已訛爲“白水”。\\
\\
解讀:\\
“忄亻”等不會對字源洄溯造成任何困難。\\
“儿”與“人”古已分化,形近義通也。\\
“⺈”表人者,“敻奐”等。然自篆書始,亦可表動物之頭部,如“象兔魚”等。\\
“阝”只要分清方位原理,可以精確溯求至“阜”或“邑”。\\
“艸”在絕大多數情況下可以理解爲草本植物,然而也有例外,如“萬”。\\
“泉”切不可解釋爲“白色的流水”,“白”在此處是不可作爲成字解讀的。\\
\\
由以上兩點可得,以楷書進行字源之洄溯,絕大部分是精準可靠的,少部分會因爲訛變而發生錯誤。對於這些可能有字義遷變合併的楷書部件,要對應返還至篆體等能體現字源的字體,再聯繫楷書分析變化,以達溯源識義之目標。\\
\\


\section{<二.倉頡字母之組合>}

在我研究傳統文化和漢字字源的近一年以來,我深深感覺到朱邦復賢者“以中華文化爲終生職志”義不容辭的深責重仼。朱邦復前輩爲其夙願,孜孜不倦地對中華文化行以研究和創作。根據其夥伴和公司員工的共述,他幾十年來未曾鬆懈一日。今朱老先生已近八十歲之高齡,子曰:“生無所息!”朱前輩如是!\\
他所取得的成就之多,世所罕見,其中關於漢文字學的研究不完全列舉如下\footnote{http://www.cbflabs.com/?id=5}:\\
漢文典庫\\
<<字易>>\\
<<詞易>>\\
<<漢字基因>>\\
<<罟網鴻爪>>、<<罟網雪泥>>(其中有無數條關於漢字的解答)\\
<<層次論>>(篇末由自然人性、社會生活引申至漢字概念的分類和歸納)\\
\\
朱邦復賢者所創造的“形音辨義”之編碼理論,“徹底解決漢字在資訊處理上的各種問題”,不僅能保全過去的所有中文字,還能利用電腦智慧創新組合。他在倉頡手冊中總結道:“最重要的是字義基因,也是自然語言的成敗關鍵”。可見倉頡輸入法追溯字源字義志向明確,勢在必得。\\
\\
朱老將說文所選出之漢字進行整理,析出594個字首,9897個字身。所有部件,不出原始象形及指事等字形。“是故這些字首及字身可以分別居碼,以代表原字,亦可按中文字形組合之規則,以字首及字身結合之。”\\
定立中文字母之時,其標準是:“在文字組合中,應用頻率高,在文義中,具有明確的定義。”輔助字形的選擇與歸類,原則爲:必須是使用頻率最高的基本元素,且與中文字母的定義相關,且避免形義的重覆。\\
最終,倉頡編碼之組合僅用114個“基本符號”,根據漢字的結構規則,即可得到該字“倉頡碼”。\\
這就是“倉頡碼”與漢字字源如此之親近,能相應互釋的根本原因。倉頡中文字母和其輔助字母,其本質就是字源字義之基因。倉頡輸入法之原名爲“形意檢字法”,形意檢字即是倉頡輸入法的初衷,這是倉頡最顯著最根本的特點。\\



\section{<三.漢字舉例與解說>}

此篇舉例,意在推出實際依據,以證倉頡自始至終未有偏離其符合字源之志。\\
由於可用例字過多,在此僅挑揀部分,以“合代斷分匹”分節。\\

\subsection{合:長出}

長\\
倉頡取形:?一?\\
倉頡字母:尸一女\\
《說文》:“長,久遠也。从兀,从匕。兀者,高遠意也。久則變化。亾聲。●者倒亾也。?,古文長。”\\
長自造字伊始便與毛髮有密切關係,說文“(卜兀)古文長”。後加彡成“髟”專門表示人之毛髮,镸同長。例字“髮鬚鬍鬢”等。\\
倉頡字母“尸”裡的輔助字形(镸上部),有表示長毛之用,例:馬[尸手尸火]。\\
次碼“一”用作指事之符,無實義,例:甘[廿一]。\\
“女”中有“止”之形,表人足,即人之下身,例:辰[一一一女]。\\
倉頡字母組合“尸一女”形成一個長髮站立之人,詡詡如生。\\

出\\
倉頡取形:屮凵\\
倉頡字母:山山\\
《說文》:“出,進也。象艸木益滋上出達也。”\\
觀其字源,上爲草木,下爲土地。\\
倉頡“山”裡有輔助字形“屮”,一直可用作初生之草,例:艸[山山]。\\
第二碼“凵”象土地也,亦包含在字母“山”中,例:凷[山土]。\\
則倉頡字母組合“山山”道出了“出”字之本義,簡明易懂。\\
\\
\subsection{代:萬泉}

萬\\
倉頡取形:艹田厶冂\\
倉頡字母:廿田戈月\\
《說文》:“萬,蟲也。从厹,象形。”\\
“萬”本爲連體象蝎子之形,倉頡字母“廿”裡有輔助字母“艸”,借形表示蟲之前螯,題意“代”也。\\
“田”在倉頡取形中,可表動物軀體,例:魚[弓田火]。\\
“戈月”組合成二一四部首“禸”,表動物之足,例:离[卜山大月]。\\
這樣,倉頡字母組合“廿田戈月”形成一只迅疾移動,蓄毒待發的蝎子。\\
值得一提的是,能從“萬”識別出二一四部首“禸”,印證了朱邦復始制倉頡之時,整理出的9897個字身,“所有部件,不出原始象形及指事等字形”。\\
\\
泉\\
倉頡取形:丿日水\\
倉頡字母:竹日水\\
《說文》:“泉,水原也。象水流出成川形。”\\
從篆體字源聯繫楷書變化,“丿”爲泉眼、泉源,“曱”象水遇石分流之形,“水”即爲“川”。分以代釋“泉”整字之各部。\\
若非以朱邦復先生“避免形義重覆”之原則,所選定字形若有獨體“白”,則成“白水爲泉”,誤矣。\\
\\
\subsection{斷:妻聿}

妻\\
倉頡取形:十肀女\\
倉頡字母:十中女\\
《說文》:“妻,婦與夫齊者也。从女,从屮,从又。又,持事,妻職也。”\\
\\
聿\\
倉頡取形:肀?\\
倉頡字母:中手\\
《說文》:“聿,所以書也。楚謂之聿,吳謂之不律,燕謂之弗。从?,一聲。”\\
\\
以上“妻聿”二字均含有“以手持物”之部件“肀”,倉頡字母“中”裡的輔助字母“肀”正能準確表示以手持物之狀。\\
若非有倉頡“斷”之取碼方法,把“十肀”、“肀?”斷開,則妻不持事,聿無以書也。\\
\\
\subsection{分:里重}

里\\
倉頡取形:田土\\
倉頡字母:田土\\
《說文》:“里,居也。从田,从土。”\\
我想“里”已經不必多言,田土僅是上下相置,其形明顯是二字之組合。惜其他輸入法定要保全一筆不二斷,無法分開,只好取作“日土”或“甲二”,毫無靈活可言。\\
\\
重\\
倉頡取形:丿十田土\\
倉頡字母:竹十田土\\
《說文》:“重,厚也。从壬,東聲。”\\
“東”貫“壬”中,楷書已難看出“壬東”之形。\\
壬:[竹土]\\
東:[木田]\\
楷書之“東”省略下方“小”的部分,倉頡儘以[十田]作東省,這樣便準確分析出“重”字之組成。其他輸入法若指形賦義,解作“千里之行,腳步沉重”,果真謬以千里。\\
\\
\subsection{匹:牛羊}

牛\\
倉頡取形:丿?\\
倉頡字母:竹手\\
《說文》:“牛,大牲也,牛件也。件,事理也。象角頭三,封尾之形。”\\
\\
羊\\
倉頡取形:卝?\\
倉頡字母:廿手\\
《說文》:“羊,祥也。从,象頭角足尾之形。孔子曰:牛羊之字,以形舉也。”\\
\\
“牛羊”二字,倉頡以丿匹配長而尖之牛角及牛頭,以卝匹配彎曲之羊角及羊頭,加以?(丯少一橫之形)表示四肢軀幹。這種匹義代換方法在倉頡之於象形字之取碼,用之頻繁。\\
\\
再例:“象魚兔”三字,倉頡把“象頭”、“魚頭”、“兔頭”均匹配至輔助字形“⺈”中,則通過承接不同的身型,得出不同的字義。簡直和倉頡造字異曲同工!\\



\section{<四.輸入法對比評說>}

當今中文輸入法,以純音檢字,有拼音或注音,使用者十之八九,不得不感歎國人學力之低,德責之薄。其音又以普通話或國語爲之,與中古音已千差萬別。長此以往,傳統國學不存自不必說,連執筆書寫之能力都將消亡褪化!\\
凡編碼中全爲音者,或涉及音者,與字源無關。\\
\\
簡化字之殘形,如“东车练鸟乌龙”等,東拐西斜,醜陋之極,剮骨歺肉,源義盡失。或有“打簡出繁”,豈料一簡多繁問題無解,諸如“洗發水”,“五裡路”,笑話百出。\\
以簡化字之偏旁作字根者,或僅以簡化字作基礎詞庫者,非與字源相關。\\
\\
更有某輸入法,也自稱與字源相關。吾觀之,不過同義之楷書成字置於一鍵,此法利於字根記憶,但與字源無關。因“字源”係一字字形之源,而非一眾字。\\
據義吞形不析構,無有字形之源,字義從何而來?\\
\\
\\
\\
\section{<結語>}

當今中文輸入法之中,惟倉頡最能符合漢字字源。\\
\end{document}
